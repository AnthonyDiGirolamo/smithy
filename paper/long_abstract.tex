\documentclass[conference]{IEEEtran}
% \hyphenation{op-tical net-works semi-conduc-tor}

\begin{document}
\title{Improving Software Installation Techniques at the\\
National Center for Computational Sciences at\\
Oak Ridge National Laboratory}

\author{\IEEEauthorblockN{Anthony DiGirolamo}
\IEEEauthorblockA{National Center for Computational Sciences\\
Oak Ridge National Laboratory\\
Oak Ridge, Tennessee\\
Email: anthonyd@ornl.gov}
\and
\IEEEauthorblockN{Robert D. French}
\IEEEauthorblockA{National Center for Computational Sciences\\
Oak Ridge National Laboratory\\
Oak Ridge, Tennessee\\
Email: frenchrd@ornl.gov}
}
\maketitle

% \begin{abstract}
% The abstract goes here.
% \end{abstract}
\IEEEpeerreviewmaketitle


\section{Introduction}

Smithy is a software compilation and installation tool that borrows ideas
heavily from SWTools\cite{swtools} and the homebrew\cite{homebrew} package
management system for Mac OS X.  Smithy is designed to manage many software
builds within an HPC Linux environment using modulefiles to load software into a
user's shell.  SWTools has set very good conventions for software installations
at the NCCS.  Smithy's goal is to make following the SWtools conventions easier
and less error prone.  Smithy improves upon SWTools by providing a simpler
command line interface, modulefile generation and management, and installations
via formulas written in Ruby.

\section{Improving on SWTools}

SWTools relied on build scripts to perform installations.  The problem we
encountered with build scripts is that they are duplicated for every software
installation.  This can make installing new software tedious since one has to go
back and look at existing build scripts and copy relevant steps to a new build
script.  Frequently build scripts simply run the default "configure, make, install"
pattern and omit custom changes to Makefiles or source code. This knowledge is
easily lost and can make future installs difficult to replicate and maintain.

\section{Installation Formulas}

The goal of installation formulas in Smithy is to consolidate all knowledge
required to build a software package on any system into a single file.  This
removes the problem of duplicate build scripts and missed steps.  Formulas can
include: defining dependencies, loading or swapping modules, setting environment
variables, applying patches, creating or changing makefiles, running the
compilation, running tests, and defining a modulefile.  It is recommended that
formulas be stored in a version control system such as git or subversion.  The
NCCS uses github.com to store a repository with all its formulas.  This allows
for easy sharing and collaboration with the HPC community.  New formulas or
changing existing ones can be sent as a pull request between HPC centers.

\section{Documentation}

Smithy provides detailed documentation in manpages and on its website.
Additionally any Smithy subcommand can be prefixed with 'help' on the command line
to show all relevant options and arguments.  There is also command line tab
completion for bash and zsh.  Documentation for installed software can be now
written in markdown or html.  Under the SWTools scheme, installing the same
package on multiple systems would require separate documentation for each
installation.  Smithy adds an option to use only one documentation file per
software package across all systems.

\section{Conclusion}

Formulas are a way to programatically install software in such a way that is
self documenting.  We have been able to use them to great effect on systems at
the NCCS.  Our hope is that Smithy software installation formulas for different
operating systems and software packages will be shared between HPC centers. This
could greatly reduce the time required to install software.


\begin{thebibliography}{1}

\bibitem{swtools}
N. Jones, M. R. Fahey, "Design, Implementation, and Experiences of Third-Party
Software Administration at the ORNL NCCS," Proceedings of the 50th Cray User
Group (CUG08), Helsinki, May 2008.

\bibitem{homebrew}
Homebrew. brew.sh. Retrieved December, 2013 from http://brew.sh/

\end{thebibliography}

\end{document}


